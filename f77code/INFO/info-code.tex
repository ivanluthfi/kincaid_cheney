%21 May 1991
\baselineskip=10pt
\parskip=10pt
\centerline{\bf           NUMERICAL MATHEMATICS AND COMPUTING, Second Edition}
\centerline{                 Ward Cheney \& David Kincaid}
\centerline{	Brooks/Cole Publishing Co.}
\centerline{                     \copyright 1985 ISBN 0-534-0456-9}
\medskip
\centerline{			May 21, 1991}
\medskip
\noindent
{\bf INTRODUCTION}:  These files of sample programs use the
subroutines developed in this textbook.  We believe that these 
computer routines are coded in a clear and easy-to-understand style.
In fact, comment lines have been intentionally omitted to force the
student to read the code and to study the textbook.  While these programs
are primarily a learning and teaching aid for use with this textbook,
they are adequate for many nonpathological problems.  They 
are usable on computing systems with Fortran 77 compilers from small
personal computers to large scientific computing machines.  These 
routines are not intented to be general-purpose software such as those
found in scientific libraries.  Nevertheless, we have found them to be
surprising useful for a variety of purposes.

\noindent
{\bf INSTALLATION}:  On a Unix system, unpack the file
cheney-kincaid.shar as follows

\indent
{\tt sh cheney-kincaid.shar}

\noindent
These programs will run as is on any computer with a 
standard Fortran 77 compiler.  However, the statement

\indent
{\tt data epsi/1.0e-6/}

\noindent
in routines {\tt rk45ad.f} and {\tt amkad.f} should be changed to the
machine epsilon (single precision roundoff error) for the 
computer that is to be used.

\noindent
{\bf USAGE}:  Compile and execute the program on file rombrg.f as follows

\indent
{\tt f77 rombrg.f}

\indent
{\tt a.out}

\noindent
{\bf AVAILABILITY}:  For information on the availability on a floppy diskette 
containing these Fortran 77 programs write either the publisher of the 
textbook or the authors at the following addresses.
\vskip .25in
$$\vbox{\halign{\indent\indent#\hfil\cr
                Brooks/Cole Publishing Co.\cr
		511 Forest Lodge Road\cr
		Pacific Grove, CA 93950--5098\cr
		(408) 373--0728\cr
		fax: (408) 375--6414\cr
		\cr
		\cr
		Center for Numerical Analysis\cr
		University of Texas at Austin\cr
		Austin, TX 78713--8510\cr
		(512) 471--1242\cr
		fax: (512) 471--9038\cr
		kincaid@cs.utexas.edu\cr}}
$$
\medskip
\noindent
This software is also available by sending email to netlib@ornl.gov
or netlib@research.att.com with a message asking for
the cheney-kincaid index and for shar-file.
In other words, by sending email to either netlib address
with the messages ``send index from cheney-kincaid and
``send shar from cheney-kincaid, one obtains the 
information sheet and the code associated with this text, respectively.


\vfill
\eject
\halign{\tt#\hfil&\hfil#&\quad#\hfil\cr
{\bf File Name}&{\bf Pages}&
{\bf Description of Code (Subprogram Names)}\cr

pi.f&4&
Simple code to illustrate programming in double precision\cr

exp.f&11-12&
First programming experiment\cr

xsinx.f&63-65&
Example of programming $f(x) = x - \sin(x)$ carefully 
{\tt (F)}\cr

bisect1.f&77-78&
First version of Bisection method 
{\tt (BISECT,F,G)}\cr

bisect2.f&77-78&
Second version of Bisection method 
{\tt (BISECT,F,G)}\cr

newton.f&85-86&
Sample Newton method program\cr

coef.f&117-118&
Newton interpolation polynomial for $sin(x)$ at equidistant points 
{\tt (COEF,EVAL)}\cr

deriv.f&141-142&
Derivative by center differences and Richardson extrapolation 
{\tt (DERIV,F)}\cr

ulsum.f&155-156&
Upper and lower sums programming experiment for an integral\cr

trap.f&161&
Trapezoid rule programming experiment for an integral\cr

rombrg.f&173-174&
Romberg arrays for three separate functions 
{\tt (ROMBRG,F,G,P)}\cr

simp.f&187-190&
Adaptive scheme for Simpson's rule 
{\tt (SIMP,ASMP,PUSH,POP,FCN)}\cr

ngauss.f&208-209&
Naive Gaussian elimination to solve linear systems 
{\tt (NGAUSS)}\cr

gauss.f&220-223&
Gaussian elimination with scaled partial pivoting 
{\tt (GAUSS,SOLVE,TSTGAUS)}\cr

triqnt.f&233-236&
Solves tridiagonal \& pentadiagonal linear systems 
{\tt (TRI,PENTA)}\cr

spl1.f&261&
Interpolates table using a first-degree spline function
{\tt (SPL1)}\cr

spl3.f&277-278&
Natural cubic spline function for sin(x) at equidistant points 
{\tt (SPL3,ZSPL3)}\cr

aspl2.f&297-298&
Interpolates table using a quadratic B-spline function
{\tt (ASPL2,BSPL2)}\cr

sch.f&300&
Interpolates table using Schoenberg's process 
{\tt (SCH,ESCH,F)}\cr

euler.f&306-307&
Euler's method for solving an ordinary differential equation\cr

taylor.f&307-308&
Taylor series method (order 4) for solving an ordinary differential equation\cr

rk4.f&315-316&
Runge-Kutta method of order 4 for solving an initial value problem 
{\tt (RK4,F)}\cr

rk45.f&326&
Runge-Kutta-Fehlberg method for solving an initial value problem 
{\tt (RK45,F)}\cr

rk45ad.f&326-328&
Adaptive scheme based on Runge-Kutta-Fehlberg method 
{\tt (RK45AD,RK45,F)}\cr

random.f&337&
Example to compute, store, and print random numbers 
{\tt (RANDOM)}\cr

testra.f&338&
Coarse check on the random-number generator 
{\tt (RANDOM)}\cr

mctst1.f&347&
Volume of a complicated region in three-space by Monte Carlo 
{\tt (RANDOM)}\cr

mctst2.f&345-346&
Numerical value of integral over a disk in xy-space by Monte Carlo 
{\tt (RANDOM)}\cr

cone.f&348&
Ice cream cone example 
{\tt (RANDOM)}\cr

sim1.f&351-352&
Loaded die problem simulation 
{\tt (RANDOM)}\cr

brthdy.f&352-354&
Birthday problem simulation 
{\tt (PROB,BRTHDY,RANDOM)}\cr

sim2.f&354-355&
Buffon's needle problem simulation 
{\tt (RANDOM)}\cr

sim3.f&355-356&
Two dice problem simulation 
{\tt (RANDOM)}\cr

sim4.f&356-358&
Neutron shielding problem simulation {\tt (RANDOM)}\cr

taysys.f&393&
Taylor series method (order 4) for system of ordinary differential equations\cr

rk4sys.f&394-396&
Runge-Kutta method of order 4 for a system of ode's 
{\tt (RK4SYS,XPSYS)}\cr

amrk.f&406-408&
Adams-Moulton method for systems of ode's 
{\tt (AMRK,RKSYS,AMSYS,XPSYS)}\cr

amrkad.f&408-409&
Adaptive scheme for Adams-Moulton method for systems of ode's\cr
&&\qquad{\tt (AMRKAD,XPSYS,AMSYS,AMRK,RKSYS)}\cr

bvp1.f&418-420&
Boundary value problem solved by discretization technique 
{\tt (TRI)}\cr

bvp2.f&421-423&
Boundary value problem solved by shooting method 
{\tt (RK4SYS,XPSYS)}\cr

pde1.f&433&
Parabolic partial differential equation problem\cr

pde2.f&434&
Parabolic pde problem solved by Crank-Nicolson method 
{\tt (TRI)}\cr

string.f&441-442&
Hyperbolic pde problem solved by discretization 
{\tt (F,TRUE)}\cr

seidel.f&449-451&
Elliptic pde solved by discretization and Gauss-Seidel method\cr
&&\qquad {\tt (SEIDEL,F,G,BNDY,USTART,TRUE)}\cr

info-code.tex&&
\TeX\  File with this information in it\cr

info-code.tty&&
File with this information in it\cr
}


\vfill
\bye
