%28 June 94
%
% Numerical Mathematics and Computing, Third Edition
% Ward Cheney & David Kincaid
% Brooks/Cole Publ. Co.
% ISBN 0-534-20112-1
% Copyright (c) 1994.  All rights reserved.
% For educational use with the Cheney-Kincaid textbook.
% Absolutely no warranty implied or expressed.
%
\documentclass{article}
\begin{document}

Software associated with the textbook 
{\em Numerical Mathematics and Computing, Third Edition},
Ward Cheney \& David Kincaid, Brooks/Cole Publishing Co.,
1994, ISBN 0--534--20112--1, is available from several sources:

\begin{itemize}
\item
a WEB browser to connect to

{\tt ftp://ftp.ma.utexas.edu/pub/papers/CNA/cheney-kincaid}

\item
by using anonymous ftp from math.utexas.edu as follows:

\begin{verse}
ftp math.utexas.edu\\
Name: anonymous\\
Password: user ID\\
ftp$>$ cd pub/papers/CNA/cheney-kincaid\\
ftp$>$ get README 
\end{verse}

\item
 by sending email to netlib@ornl.gov
with the message {\em send index from cheney-kincaid}.
\end{itemize}

Files are available for the pseudocode in the textbook
written in either Fortran 77, Fortran 90, or C.
Also, files  are available containing Maple and Matlab programs.

The table on the following pages lists Fortran 90 
files containing sample programs 
based on the pseudocode given in the book.
They are intended  primarily as a learning and teaching aid 
for use with this book.
We believe that these computer routines are coded in a clear and
easy-to-understand style but we
have intentionally excluded comment statements 
so that students will read the code and study the algorithms---they
can add comments as they decipher them.
These programs are usable on computer systems with appropriate 
compilers, from
small personal computers to large scientific computing machines.
However, they do not contain all of the ``bells-and-whistles'' of
robust state-of-the-art software such as may be found in general-purpose
scientific libraries.
Nevertheless, they are adequate for many small nonpathological
problems.

\begin{center}
\begin{tabular}{l@{\quad}r@{\qquad}l}
{\bf File Name} & {\bf Pages} 
&{\bf Description of Fortran 90 Code} \\[0.1in]
CHP1/\\
\quad pi.f90&7&Simple code to illustrate double precision\cr
\quad first.f90&10--11&First programming experiment\cr
CHP2/\\
\quad numbers.f90&53--55&Print internal machine representation of various numbers\\
\quad hex.f90&60--62&Print in hexadecimal format\\
\quad oct.f90&60--62&Print in octal format\\
\quad xsinx.f90&71&Example of programming $f(x) = x - \sin x$ carefully\\
CHP3/\\
\quad bisection.f90&85&Bisection method\\
%\quad bisection2.f90&85&Second version of bisection method\\
\quad rec\_bisection.f90&86&Recursive version of bisection method\\
\quad newton.f90&95&Sample Newton method\cr
\quad secant.f90&110&Secant method\cr
CHP4/\\
\quad coef.f90&132&Newton interpolation polynomial at equidistant pts\\
\quad deriv.f90&156&Derivative by center diff./Richardson extrapolation\\
CHP5/\\
\quad sums.f90&171&Upper/lower sums experiment for an integral\cr
\quad trapezoid.f90&176&Trapezoid rule experiment for an integral\cr
\quad romberg.f90&190&Romberg arrays for three separate functions\\
\quad rec\_simpson.f90&204&Adaptive scheme for Simpson's rule\\
CHP6/\\
\quad ngauss.f90&224&Naive Gaussian elimination to solve linear systems\\
\quad gauss.f90&236/239&Gaussian elimination with scaled partial pivoting\\
\quad tri.f90&251&Solves tridiagonal systems\\
\quad penta.f90&253&Solves pentadiagonal linear systems\\
CHP7/\\
\quad spline1.f90&281&Interpolates table using a first-degree spline function\\
\quad spline3.f90&297&Natural cubic spline function at equidistant points\\
\quad bspline2.f90&319&Interpolates table using a quadratic B-spline function\\
\quad sch.f90&322&Interpolates table using Schoenberg's process\\
CHP8/\\
\quad euler.f90&328&Euler's method for solving an ODE\cr
\quad taylor.f90&319&Taylor series method (order 4) for solving an ODE\cr
\quad rk4.f90&339&Runge-Kutta method (order 4) for solving an IVP\\
\quad rk45.f90&351&Runge-Kutta-Fehlberg method for solving an IVP\\
\quad rk45ad.f90&352&Adaptive Runge-Kutta-Fehlberg method\\
CHP9/\\
\quad taylorsys1.f90&362&Taylor series method (order 4) for systems of ODEs\cr
\quad taylorsys2.f90&364&Taylor series method (order 4) for systems of ODEs\cr
\quad rk4sys.f90&365&Runge-Kutta method (order 4) for systems of ODEs\\
\quad amrk.f90&377--378&Adams-Moulton method for systems of ODEs\\
\quad amrkad.f90&408--409&Adaptive Adams-Moulton method for systems of ODE's\\
\end{tabular}
\end{center}

\newpage

\begin{center}
\begin{tabular}{l@{\quad}r@{\qquad}l}
{\bf File Name} & {\bf Pages} 
&{\bf Description of Fortran 90 Code} \\[0.1in]
CHP11/\\
\quad test\_random.f90&413/414&Example to compute, store, and print random numbers\\
\quad coarse\_check.f90&414&Coarse check on the random-number generator\\
%ellipse.f&416&\cr
%ellipse-erroneous.f&417&\cr
%circle-erroneous.f&418&\cr
\quad double\_integral.f90&424&Volume of a complicated 3D region by Monte Carlo\\
\quad volume\_region.f90&345--346&Numerical value of integral over a 2D disk by Monte Carlo\\
\quad cone.f90&426&Ice cream cone example\\
\quad loaded\_die.f90&430&Loaded die problem simulation\\
\quad birthday.f90&432&Birthday problem simulation\\
\quad needle.f90&433&Buffon's needle problem simulation\\
\quad two\_die.f90&434&Two dice problem simulation\\
\quad shielding.f90&435&Neutron shielding problem simulation\\
CHP12/\\
\quad bvp1.f90&450&Boundary value problem solved by discretization technique\\
\quad bvp2.f90&453&Boundary value problem solved by shooting method\\
CHP13/\\
\quad parabolic1.f90&462&Parabolic partial differential equation problem\cr
\quad parabolic2.f90&464&Parabolic PDE problem solved by Crank-Nicolson method\\
\quad hyperbolic.f90&472&Hyperbolic PDE problem solved by discretization\\
\quad seidel.f90&480&Elliptic PDE solved by discretization/ Gauss-Seidel 
method\\
INFO/\\
\quad info\_f90\_code.ps&&Postscript file containing information on f90 codes\cr
\quad info\_f90\_code.tex&&\LaTeX\  file containing information on f90 codes\cr
\quad info\_f90\_code.tty&&ASCII file containing information on f90 codes
\end{tabular}
\end{center}

\end{document}
